\subsection{Horizontális értelmezés}
\label{sec:horizontalisertelmezes}
A hangok illetve akkordok hangemhez kötött, horizontális értelmezéséve szerkesztett dallammenet csakis a skála hangjait tartalmazza. Ennek módosított - \index{alteráció}alterált -, skálával való színezésére az alább ismertetett módszerek használatosak.
\subsection{Vertikális értelmezés}
\label{sec:vertikalisertelmezes}
A hangok illetve akkordok hangemtől független értelmezése. A zenemű színezése történhet oly módon, hogy az éppen kísérő akkordhoz nem a hangemnek megfelelő dallamot szerkesztünk. Ehhez egy, vagy több olyan skála hangjait kell felhasználni, amelyben az adott kísérő akkord szerepel. Lásd a ,,\nameref{sec:exdallamvert}'' példát.
\subsection{Enharmonikus kiegészítés}
\label{sec:enhkiegeszites}
A kíslerő akkord enharmonikus akkordjaihoz tartozó skálák összevonásával készült alterált skála felhasználható a dallammenet színezésére. Enharmonikus akkordok, melyeknek a kísérő akkorddal legalább két közös hangjuk van, a következő elvek alapján találhatók: \\\\
A kísérő akkord hangjaiból álló kisebb fokszámú hangzat választása (pl. négyesből hármas). \\
A kísérő akkord legalább két hangjánkak kiegészítése olyan hangokkal, melyeket az nem tartalmaz. \\
A kísérő akkordban, vagy az első elv szerint tartalmazott kisebb fokszámú hanzatban egyes hangok megváltoztatása. \\\\
Lásd a ,,\nameref{sec:exdallamenharm}'' példát.
\subsection{Hangcsoportok}
\label{sec:hangcsoportokII}
A dallam hangjait három csoportba soroljuk.
\begin{pitemize}
I & Pillér hangok & a kísérő akkord hangjai \\
II & Diatonikus váltóhangok & a skála hangjai, kivéve a pillérhangokat \\
III & Kromatikus váltóhangok & a kísérő akkord hangjaitól legfeljebb szekund távolságra lévő hangok, \\
                       && melyek nem szerepelnek a skálában \\
\end{pitemize}
Az ezekből a hangcsoportokból szerkesztett dallamnál fontos, hogy az első és harmadik negyed mindig pillér hanggal kezdődjön. A kromatikus váltóhangokat kizárólag pillérhang előtt - annak bevezetéseként - szabad használni.

