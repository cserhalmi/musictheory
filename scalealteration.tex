Ha a dallam hangjait és a kísérő akkordokat Horizontális értelmezés.
A hangok illetve akkordok hangemhez kötött, horizontális értelmezéséve szerkesztett dallammenet a skála alaphangjait tartalmazza.
\subsection{Vertikális értelmezés}
A hangok illetve akkordok hangemtől független értelmezése. A zenemű színezése történhet oly módon, hogy az éppen kísérő akkordhoz nem a hangemnek megfelelő dallamot szerkesztünk. Ehhez egy, vagy több olyan skála hangjait kell felhasználni, amelyben az adott kísérő akkord szerepel. Az így kialakított hangsort alterált skálának nevezzük.
A dallam hangjait három csoportba soroljuk.
\begin{pitemize}
Pillér hangok -- a kísérő akkord hangjai \\
Diatonikus váltóhangok -- az alterált skála hangjai, kivéve a pillér hangok \\
Kromatikus váltóhangok -- a kísérő akkord hangjaitól legfeljebb szekund távolságra lévő hangok, amelyek nem szerepelnek a hangnemben \\
\end{pitemize}
Az ezekből a hangcsoportokból szerkesztett dallamnál alapszabály, hogy az első és harmadik negyed mindig pillér hanggal kezdőjön. A kromatikus vezetőhangokat pillér hang előtt - annak bevezetéseként érdemes használni.
