\subsection{Horizontális értelmezés}
\label{sec:horizontalisertelmezes}
A hangok illetve akkordok hangemhez kötött, horizontális értelmezéséve szerkesztett dallammenet csakis a skála hangjait tartalmazza. Ennek módosított - alterált -, skálával való színezésére az alább ismertetett módszerek használatosak.
\subsection{Vertikális értelmezés}
\label{sec:vertikalisertelmezes}
A hangok illetve akkordok hangemtől független értelmezése. A zenemű színezése történhet oly módon, hogy az éppen kísérő akkordhoz nem a hangemnek megfelelő dallamot szerkesztünk. Ehhez egy, vagy több olyan skála hangjait kell felhasználni, amelyben az adott kísérő akkord szerepel. Lásd a ,,\nameref{sec:exdallamvert}'' példát.
\subsection{Enharmonikus kiegészítés}
\label{sec:enhkiegeszites}
\begin{pitemize}
A kísérő akkord hangjaiból álló kisebb fokszámú hangzat választása \\
A kísérő akkord kiegészítése \\
A kísérő akkorddal enharmonikus akkordban hangok cseréje \\
\end{pitemize}
\subsection{Hangcsoportok}
%\label{sec:hangcsoportok}
A dallam hangjait három csoportba soroljuk.
\begin{pitemize}
Pillér hangok -- a kísérő akkord hangjai \\
Diatonikus váltóhangok -- az alterált skála hangjai, kivéve a pillér hangok \\
Kromatikus váltóhangok -- a kísérő akkord hangjaitól legfeljebb szekund távolságra lévő hangok, amelyek nem szerepelnek a hangnemben \\
\end{pitemize}
Az ezekből a hangcsoportokból szerkesztett dallamnál alapszabály, hogy az első és harmadik negyed mindig pillér hanggal kezdőjön. A kromatikus vezetőhangokat pillér hang előtt - annak bevezetéseként érdemes használni.

