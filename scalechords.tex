A skála fokaira épülő akkordok meghatározásához meg kell vizsgálni, hogy az adott skála fokról kezdve milyen terc, kvint, szeptim, nóna, undecima és tredecima építhető egymásra. A ,,\nameref{sec:skalaakkordok}'' melléklet táblázatai alapján a hármas- és négyes hangzatok típusai, megjegyezhető formában:
\begin{pitemize}
Hangnem & Hangzat & I & II & III & IV & V & VI & VII \\ \hline
Dúr & Hármas & dúr & moll & moll & dúr & dúr & moll & szűk \\
Dúr & Négyes & maj7 & moll7 & moll7 & maj7 & dom7 & moll7 & félszűk7 \\
Moll & Hármas & moll & szűk & bőv & moll & dúr & dúr & szűk \\
Moll & Négyes & mollmaj7 & moll7 & félszűk7 & moll7 & dom7 & maj7 & szűk7 \\
Harmonikus moll & Négyes & moll & moll & bőv & dúr & dúr & szűk & szűk \\
Harmonikus moll & Hármas & mollmaj7 & moll7 & moll7 & dom7 & dom7 & félszűk7 & félszűk7
\end{pitemize}%
\captionof{figure}{A skálák fokairól induló akkord típusok} 
\label{fig:skalafokakkord}
\ideasubsection{Modális skálák akkordjai}
Amint az a \ref{fig:frigskalahangzatok}. táblázatban látszik, az egyes akkordtípusok megegyeznek a dúr skála megfelelő modális fokának akkordtípusaival. Így akárcsak a skálatörvényeket, az akkord típusokat is könnyen származtathatjuk a dúr skálából.
\begin{practices}
\item Sorold fel a hangnem meghatározó skálák fokairól induló hármas- és négyeshanzatok típusait!
\item Egy kiválaszott dúr, moll, vagy melodikus moll skálához sorold fel a fokairól induló hármas- és négyeshangzatokat!
\end{practices}
