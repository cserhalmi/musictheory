Az akkord egyidőben megszólaló legaláb három különböző hang.
Nem számít különböző hangnak a hangok bármelyikétől egy, vagy több oktáv távolságra lévő hang.
Két hang nem akkord, csak hangköz. Egy adott skálához tartozónak tekintjük a skála hangaiból felépített hangzatokat. Ezen akkordok típusát (ld. lent) a skála hangközei adják.

\subsection{Hármashangzatok}
A hármashangzatok olyan akkordok, melyek az alaphangra épülő terc és kvint hangközből állnak.
\begin{pitemize}
Név & Jelölés & Példa \\ \hline
C dúr       & $C$      & C E G \\
C bővített  & $C^+$    & C E \gisz \\
C moll      & $C_m$    & C \disz G \\
C szűkített & $C^\dim$ & C \disz \fisz \\
\end{pitemize}                                                                                                                                  

\ideasubsection{Az akkordok hangjainak gyors meghatározása}
Egyszerű hármashangzat hangjai könnyen felsorolhatók, ha a két tercet fejben felsoroljuk: 
C, D, E - E, F, G. Ez alapján a C-dúr akkord a C, E és G hangokból áll. 
Ha a alaphang félhang, akkor is érdemes a közeli egész hanggal számolni, majd mindhárom
hangot eltolni fél hanggal a megfelelő irányba.

\begin{practices}
\item Definiáld a hármashangzat fogalmát, sorold fel fajtáit, a fajták felépítését és jelölését!
\item Sorold fel tetszőleges alaphanggal egy moll-, szűkített-, bővített- és dúr hármashangzat hangjait!
\item Sorolj fel a kromatikus skála bármely hangján kezdődő dúr-, moll- és melodikus moll hangsorok fokaira épülő hármashangzatokat!
\item Határozz meg hármashangzatokat a hangjai alapján!
\end{practices}

\subsection{Négyeshangzatok}
A négyeshangzatok olyan akkordok, melyeben a hármashangzatra szeptim hangköz épül.
Mivel a szeptim az akkord skálájának hangajai között a hetedik, a négyeshangzatokat hetes akkordoknak nevezzük.
\begin{pitemize}
Név & Jelölés & Példa \\ \hline
C domináns hetes            & $C^7$           &   C E G \aisz \\
C major hetes               & $C^{maj7}$      &   C E G H \\
C moll hetes                & $C_m^7$         &   C \disz G \aisz \\
C moll major hetes          & $C_m/^{maj7}$   &   C \disz G H \\
C bővített major hetes      & $C^{maj7(5\#)}$ &   C E \gisz H \\
C félszűkített hetes        & $C^\sdim$       &   C \disz \fisz \aisz \\
C szűkített hetes           & $C^\dim$        &   C \disz \fisz A \\
\end{pitemize}                                                                                                                                  
\begin{practices}
\item Definiáld a négyeshangzat fogalmát, sorold fel fajtáit, a fajták felépítését és jelölését !
\item Sorold fel tetszőleges alaphanggal az ismert négyeshangzat fajták hangjait!
\item Sorolj fel a kromatikus skála bármely hangján kezdődő dúr-, moll- és melodikus moll hangsorok fokaira épülő négyeshangzatokat!
\item Határozz meg négyeshangzatokat a hangjai alapján!
\end{practices}

\subsection{Ötöshangzatok}
Az ötöshangzatok olyan akkordok, melyeben a négyeshangzatra oktávon túli szekund, azaz nóna hangköz épül.
Mivel a nóna az akkord skálájának hangjai között a kilencedik, az ötöshangzatokat kilences akkordoknak nevezzük.
Bővített kilences hangzat nem létezik moll, szűkített illetve félszűkített hetes alapon, mert megsértené a hangzat
hangjainak különbözőségére vonatkozó szabályt. Az alapként szolgáló négyeshangzat jelöléséből kimaradhat a hetes szám,
hiszen szeptim nélkül nem beszélhetnénk ötös hangzatról.
\begin{pitemize}
Név & Jelölés & Példa \\ \hline
C kilences                     & $C^{9}$          & C E G \aisz D \\
C bővített kilences            & $C^{9\#}$        & C E G \aisz \disz \\
C szűkített kilences           & $C^{9b}$         & C E G \aisz \cisz \\
C major kilences               & $C^{maj9}$       & C E G H D \\
C moll major kilences          & $C_m/^{maj9}$    & C \disz G \aisz D \\
C bővített major szűk kilences & $C^{maj9b(5\#)}$ & C F G H \cisz \\
...
\end{pitemize}

\subsection{Hatoshangzatok}
Az hatoshangzatok olyan akkordok, melyeben az ötöshangzatra oktávon túli kvart, azaz undecima hangköz épül.
Mivel a undecima az akkord skálájának hangjai között a tizenegyedik, az hatoshangzatokat tizenegyes akkordoknak nevezzük.
A dúr hármashangzatra épülő hatoshangzat jelölése rendhagyó - a tiszta kvart bővítés jelölése $11b$, a bővített kvarté $11$.
Bővített undecimát tartalmazó hatoshangzat nem létezik , szűkített szeptim alapon, mert megsértené a hangzat hangjainak 
különbözőségére vonatkozó szabályt. Néhány példa:
\begin{pitemize}
Név & Jelölés & Példa \\ \hline
C szűkített tizenegyes      & $C^{11b}$        & C E G \aisz D E \\
C tizenegyes                & $C^{11}$         & C E G \aisz D F \\
C moll szűkített tizenegyes & $C_m^{11b}$      & C \disz G \aisz D E \\
C moll major tizenegyes     & $C_m/^{maj11}$   & C \disz G H D F \\         
C bővített major tizenegyes & $C^{maj11(5\#)}$ & C E \gisz H D F \\
...         
\end{pitemize}

\subsection{Heteshangzatok}
Az heteshangzatok olyan akkordok, melyeben az hatoshangzatra oktávon túli szext, azaz tredecima hangköz épül.
Mivel a tredecima az akkord skálájának hangjai között a tizenharmadik, az heteshangzatokat tizenhármas akkordoknak nevezzük.
Szűkített tredecimát tartalmazó hatoshangzat csak szűkített szeptim alapon létezik.
Bővített tredecimát tartalmazó hatoshangzat csak nem bővített kvint és nagyszeptim alapon létezik.
Tiszta tredecimát tartalmazó hatoshangzat csak nagyszeptim alapon létezik. \\\\
A jelölés nem tér el az eddigiektől, az alap hatos hangzat nevéhez 13, 13b, vagz 13\# index adódik.

\subsection{Suspend akkordok}
Az alaphangra épülő tiszta kvartot és kvintet tartalmazó akkordok. 
A suspend akkord után minden esetben a hangnem tercet tartalmazó akkordja következik.
Jelölése: $C^4$

\subsection{Additional akkordok}
Olyan akkordok, melyekben a hangzatra egy nem szomszédos hangzat hangköze épül.
Hármashangzatra így nóna, undecima és tredecima épülhet az ötös-, hatos- és heteshangzatokból.
Jelölése: $C^{add7}$

\subsection{6-os akkordok}
A hatos akkordok olyan négyeshangzatok, melyekben a hármashangzatra nagy szext hangköz épül.
Jelölése és fajtái:
\begin{pitemize}
Név & Jelölés & Példa \\ \hline
C dúr hatos   & $C^6$   & C E G A \\
C moll hatos  & $C_m^6$ & C \disz G A \\
\end{pitemize}

\subsection{6/9-es akkordok}
A 6/9-es akkordok olyan négyeshangzatok, melyekben a hatos akkordra nagy nóna hangköz épül.
Jelölése és fajtái:
\begin{pitemize}
Név & Jelölés & Példa \\ \hline
C dúr 6/9  & $C^{6/9}$   & C E G A D \\
C moll 6/9 & $C_m^{6/9}$ & C \disz G A D \\
\end{pitemize}

\subsection{9/6-os akkordok}
A 9/6-es akkordok olyan hagyományos ötös hangzat-beli kilences akkordok, melyekhez díszítésképpen
egy nagy szext hangköz járul.
Jelölése és fajtái:
\begin{pitemize}
Név & Jelölés & Példa \\ \hline
C 9/6      & $C^{9/6}$    & C E G aisz D A \\
C moll 9/6 & $C_m^{9/6}$  & C disz G aisz D A \\
C dúr 9/6  & $C^{maj9/6}$ & C E G H D A \\
\end{pitemize}
