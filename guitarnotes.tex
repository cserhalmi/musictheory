\subsection{Zenei hangok}
\label{sec:zeneihangok}
A zenei hang sajátossága, hogy az alaphang rezgéséhez annak felharmonikusai társulnak. Ezek összetétele határozza meg a hang színezetét. Agyunk számára olyan zenei hangok együtt-, vagy egymás után hangzása dolgozható fel kényelmesen, illetve hangzik kellemesnek, melyek felharmonikusai közel vannak egymáshoz. Bizonyíthatóan ilyenek a 6/5 (kis terc), 5/4 (nagy terc), 4/3 (tiszta kvart) illetve 3/2 (tiszta kvint) arányok. \\\\
Ebből kiindulva egy oktávnyi hangterjedelem tizenkét részre osztása terjedt el az európai zenében.
Az így felépített tiszta hangsor egymást követő elemeinek aránya nem azonos. A gitár szerkezetéből adódóan ezek az arányok enyhén eltorzultak oly módon, hogy a szomszédos osztások frekvencia viszonya $f_{h+1}/f_{h}=\sqrt[12]{2}$. Az így felépülő sort temperált (kiegyenlített) kromatikus hangsornak hívják. \\\\
A $440Hz$-es normál \key{A} hangtól indulva a kromatikus skála hangjait a
következőképpen nevezzük: \key{A}, aisz (\key{As}), \key{H}, \key{C},
cisz, \key{D}, disz, \key{E}, \key{F}, fisz, \key{G},
gisz, \key{A} vagy \key{A}, asz (\key{Ab}), \key{G},
gesz, \key{F}, \key{E}, esz, \key{D}, desz, \key{C},
\key{H}, bebé, \key{A} - attól függően, hogy növekvő-, vagy csökkenő
hangmagasság irányában haladunk. Oktávon túl a betűzés ismétlődik, ezeket a
hangokat vesszőkkel szokás megkülönböztetni.
\subsection{Hangközök, fordított hangközök}
\label{sec:hangkozok}
Az egyszerűség kedvéért az egymást követő  hangok távolságáról szoktunk beszélni. Ebben az értelemben egy félhang távolságnak tekintjük, ha az arány közel van a $\sqrt[12]{2}$-höz. Egy hanghoz képest egy másik távolságát görög sorszámnevekkel az \ref{tab:hangkozok}. táblázatban látható módon jelöljük. Módosító jelekkel a ,,tiszta'', vagy ,,nagy'' hangközökből bővített, vagy szűkített hangközöket képezhetünk. \\\\
Általában igaz, hogy ha a hangsorban felfelé haladva $h1$ nevű hang $h2$-től  $T$ hangköznyire helyezkedik el, akkor $h2$ és az azt követő $h1$ között $12-T$ távolság van. Ezt beláthatjuk, ha a hangok neveit egy körre rajzolva képzeljük el. Arra a kérdésre tehát, hogy mely hangtól van $h2$ $T$ távolságra úgy is válaszolhatunk, hogy a $h2$-től $12-T$ közre lévő hangtól. Például a hang, aminek tiszta kvartja \fisz az a \fisz hang tiszta kvintje. A fordított hangközök az \ref{tab:hangkozok}. táblázat utolsó oszlopában láthatók.
\begin{pitemize}
 hangköz & $\frac{f}{f_{prim}}$ & név & jel & példa (C-től) & fordított hangköz (C-ig) \\ \hline
 $0$ &                   & prím                              & p     & \note{1}   &                 \\
 $\frac{1}{2}$ & $2^\frac{1}{12}$  & kisszekund                        & k2    & \note{2}   & nagyszeptim     \\ 
 $1$ & $2^\frac{2}{12}$  & nagyszekund                       & n2    & \note{3}   & kisszeptim      \\
 $1\frac{1}{2}$ & $2^\frac{3}{12}$  & kisterc                           & k3    & \note{4}   & nagyszext       \\
 $2$ & $2^\frac{4}{12}$  & nagyterc                          & n3    & \note{5}   & kisszext        \\
 $2\frac{1}{2}$ & $2^\frac{5}{12}$  & tiszta kvart                      & t4    & \note{6}   & tiszta kvint    \\
 $3$ & $2^\frac{6}{12}$  & szűkített kvint - bővített kvart  & s5/b4 & \note{7}   & szűkített kvint \\
 $3\frac{1}{2}$ & $2^\frac{7}{12}$  & tiszta kvint                      & t5    & \note{8}   & tiszta kvart    \\
 $4$ & $2^\frac{8}{12}$  & kisszext                          & k6    & \note{9}   & nagyterc        \\
 $4\frac{1}{2}$ & $2^\frac{9}{12}$  & nagyszext                         & n6    & \note{10}  & kisterc         \\
 $5$ & $2^\frac{10}{12}$ & kiszszeptim                       & k7    & \note{11}  & nagyszekund     \\
 $5\frac{1}{2}$ & $2^\frac{11}{12}$ & nagyszeptim                       & n7    & \note{12}  & kisszekund      \\
 $6$ & $2^\frac{12}{12}$ & oktáv                             & to    & \note{13}' \\
 $6\frac{1}{2}$ & $2^\frac{13}{12}$ & kis nóna                          & k9    & \note{14}' \\
 $7$ & $2^\frac{14}{12}$ & nagy nóna                         & n9    & \note{15}' \\
 $7\frac{1}{2}$ & $2^\frac{15}{12}$ & kis decima                        & k10   & \note{16}' \\
 $8$ & $2^\frac{16}{12}$ & nagy decima                       & n10   & \note{17}' \\
 $8\frac{1}{2}$ & $2^\frac{17}{12}$ & tiszta undecima                   & t11   & \note{18}' \\
 $9$ & $2^\frac{18}{12}$ & bővített undecima                 & b11   & \note{19}' \\
 $9\frac{1}{2}$ & $2^\frac{18}{12}$ & duodecima                         & 12    & \note{20}' \\
 $10$ & $2^\frac{18}{12}$ & kis tredecima                     & k13   & \note{21}' \\
 $10\frac{1}{2}$ & $2^\frac{18}{12}$ & nagy tredecima                    & n13   & \note{22}' \\
\end{pitemize}
\captionof{table}{Hangközök elnevezése, jelölése és fordítása} 
\label{tab:hangkozok}
\subsection{A gitár fogólapja}
\label{sec:gitarfogolap}
A hathúros gitár húrjai, a leggyakoribb hangolással, felülről lefelé \key{E}, \key{A}, \key{D}, \key{G}, \key{H} és \key{E} hangokon szólnak. A negyedik és ötödik húr között nagy terc, míg a többi, egymást követő húr között tiszta kvart hangköz van. Mivel minden lefogás fél hang lépést jelent, így a tizenkettedik pozícióban ismét \key{E}, \key{A}, \key{D}, \key{G}, \key{H} és \key{E} hangokat találunk. Az Antonio de Torres Jurado által a XIX. században rendszeresített klasszikus- és flamenco gitár fogólapja az \ref{fig:fogolap}. ábrán látható.\\\\
\fretboard{A klasszikus gitár fogólapja}{fig:fogolap}
          {xx.x.x.xx.x.xx.x%
           xx.x.xx.x.x.xx.x%
           x.x.xx.x.xx.x.x.%
           x.xx.x.x.xx.x.xx%
           x.xx.x.xx.x.x.xx%
           xx.x.x.xx.x.xx.x}
Az egyes lefogásokhoz tartozó hangok neve (nem a magassága) könnyen meghatározható néhány szabály alkalmazásával:
\begin{pitemize}
$-$ az ötödik, tizedik és tizenkettedik lefogásnál egész hangokat találunk  \\
$-$ az ötödik lefogás a következő húr kezdőhangja, kivéve \key{G} húrt, ahol ez fél hanggal magasabb  \\
$-$ a hetedik lefogás az előző húr kezdőhangja, kivéve a \key{H} húrt, ahol ez fél hanggal alacsonyabb \\
$-$ egy húron a tizenkettedik lefogás egy oktávnyira van, ezért a hangok neve itt a húr kezdőhangja \\
$-$ a tizedik lefogáson a húr kiszeptimje van azaz egy egész hangot kell kivonni \\
$-$ a tizenkettedik lefogás felett tizenkettő kivonásával a fenti szabályok alkalmazhatók \\
\end{pitemize}
\ideasubsection{A hangközök gyors meghatározása}
A gitár hangolása gyorsan megjegyezhető. Ebből a kvart - kvint távolságok két alaphang (\key{F}, \key{C}) kivételével azonnal kiadódnak. Pl. \key{A} az \key{E} hang kvartja, ezért \key{E} az \key{A} hang kvintje. \key{G} kvartja nem a \key{H}, hanem a \key{C}, viszont a \key{H} kvartja az \key{E}. Nincs \key{F} és \key{C} húr, de ezek fél hangra vannak az \key{E} illetve a \key{H} hangoktól, tehát kvartjaik az \key{As} illetve az \key{F}. \\\\
A szekundok és tercek megtalálása nyilván nem probléma, ugyanígy - visszafelé számolva -, a szeptimek és szextek is egyszerűen kiadódnak. Később, az akkordok felépítésénél illetve a skála fokainak zenei funkció szerinti beosztásánál ennek jó hasznát vesszük.
\begin{practices}
\item Nevezz meg egy húrt és egy lefogást, majd az ott található hang nevét!
\item Tetszőleges hanghoz találd meg a kvart- és kvint hangközöket!
\end{practices}
