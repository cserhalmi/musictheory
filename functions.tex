Minden, egy adott hangnemen belül megszólaló hang, vagy akkord a hallgatóra hatást gyakorol. Ennek ismeretében szerkeszthető meg a zeneművek felépítése és lezárása.
\subsection{Fő akkordok, mellékakkordok}
\label{sec:foakkord}
A skála fokainak funkcióelméleti megnevezése sorra \index{tonika}tonika (I), szupertonika, mediáns, \index{szubdomináns}szubdomináns (IV), \index{domináns}domináns (V), szubmediáns és vezető hang.
A hangok és akkordok a hallgatóra három jellemző hatást gyakorolhatnak. Az első fokokon található tonika megnyugvó, a negyedik fokon a szubdomináns feszültségkeltő, a domináns pedig kirobbanó hatású.
Ezek a skála \textbf{fő akkordjai}, vagy hangjai.
A keltett érzet szempontjából ezek a ,,tercrokonság elve'' alapján helyettesíthetők olyan \textbf{mellék akkordokkal}, melyekkel két hangjuk közös. 
Ezek alapján a dúr-, moll-, és melodikus moll hangnemekben a fokok szerepe a következő táblázatban foglalható össze:
\begin{pitemize}
Funkció & A skála foka \\
\hline
Tonika & I tonika & III mediáns & VI szubmediáns \\
Domináns & III mediáns & V domináns & VII vezető \\
Szubdomináns & II szupertonika & IV szubdomináns & VI szubmediáns \\
\end{pitemize}
Négyeshangzat felett is érvényesül a két közös hang hatása, de ezt nem tercrokonságnak, hanem enharmóniának nevezzük: ha két akkordban kettő, vagy több hang közös, ezek hasonló hangzásúak - enharmonikusak. Ezen akkordok skálái és bontásai oda-vissza egymásra játszhatóak. \\\\
Dúr-, moll-, illetve melodikus moll skáláknál a szubdomináns és domináns akkordok, vagy fokok 
rendre tiszta kvart és tiszta kvint távolságra vannak a tonikától. Mivel azonban a különböző skálák fő akkordjai nem egy típusúak, a három fő akkord felismerésével a zenemű hangneme meghatározható. A fokok zenei funkciói egyéb skáláknál is értelmezve vannak, de a az előbbi megállapítás ezeknél általában nem igaz. \\\\
\begin{practices}
\item Határozd meg tetszőleges dúr-, vagy moll skálák fő akkordjait!
\item Határozd meg egy hármas-, vagy négyeshangzat előfordulási helyeit a hangnem meghatározó
skálákban és az ott betöltött szerepüket (pl. G dúr a C dúr domináns akkordja).
\end{practices}
\subsection{Zárlatok}
\label{sec:zarlatok}
Autentikus zárlat: I-IV-V-I. \\
Plagális (egyházi) zárlat: I-V-IV-I. \\
Félzárlat: \\
Álzárlat: V-IV \\
Andalúz zárlat: \\
dúrban : vi - V - IV - III, mollban i – bVII – bVI – V \\
http://en.wikipedia.org/wiki/Andalusian\_cadence
