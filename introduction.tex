A következő fejezetekben zenetanulmányaimat foglalom össze a lehető legtömörebben, hozzátéve pár ötletet, melyekkel a saját dolgomat próbáltam megkönnyíteni. A fejezetek elméleti fejtegetéseit a mellékelt példák és kották teszik szemléletessé és gyakorolhatóvá. Ezek mellett feljegyeztem néhány, a begyakorlást segítő feladat is. \\\\
Ez a jegyzet nem egy teljes zeneelméleti mű, inkább a megtanultak rendszerezését és ismétlését szolgáló tudástár. Természetesen alkalmas bizonyos fogalmak tisztázására, elvek megértésére, mégis azt javaslom, aki a témában el szeretne mélyülni, valamilyen formában állítson elő magának ehhez hasonló összefoglalót.
