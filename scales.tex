A zenei hangokat növekvő hangmagasság szerint rendezve skálát kapunk. Tizenkét hangot véve a lehetséges kettő, vagy több hangból álló skálák száma $2^{12} - 13 = 4083$. A hangközök sorozata adja az úgynevezett skálatörvényeket, míg a hangok sorozata a skála fokait. A hangsorokat a következőképpen szokás csoportosítani:
\begin{pitemize}
diatonikus & a hangok között nincs szekundnál nagyobb távolság \\
\hline
bichord & két fokú \\
trichord & három fokú \\
tetrachord & négy fokú \\
pentachord & öt tfokú \\
hexachord & hat fokú \\
heptachord & hét fokú (pl. a dúr skála) \\\\
nem diatonikus & a hangok között szekundnál nagyobb távolság van \\
\hline
biton & két fokú \\
triton & három fokú \\
tetraton & négy fokú \\
pentaton & öt fokú (pl. a blues skála) \\
hexaton & hat fokú \\
heptaton & hét fokú  (pl. a harmonikus moll skála)\\\\
modális & a dúr skála fokain alapuló heptachord \\
\hline
ión & dúr I fok \\
dór & dúr II fok \\
fríg & dúr III fok \\
líd & dúr IV fok \\
mixolíd & dúr V fok \\
aeol & dúr VI fok (természetes moll) \\
lokriszi & dúr VII fok \\\\
tonális & a dúr skála klasszikus módosításain alapuló \\
\hline
hangnem meghatározó skálák & dúr, (harmonikus) moll és melodikus moll \\
egyéb, nevezetes skálák & spanyol moll, magyar moll, ... \\
alterált skálák & az \index{alteráció}alteráció szabályai szerint alkotott szinező skálák \\
\end{pitemize}
\subsection{Hangnem meghatározó skálák}
Az kromatikus skálából hét elemet kiválasztva az európai zenében három különböző
hangulatú ,,tonális'' sort - hangnemet -, építettek fel, melyek skálatörvényei:
\begin{pitemize}
& I & & II & & III & & & IV & & V & & VI & & & VII & \\ \hline
dúr törvény & \tiny C & 1 & \tiny D & 1 & & \tiny E &\sst & \tiny F & 1 & \tiny G & 1 & & \tiny A & 1 & \tiny H &\sst & \tiny C \\
(harmonikus) moll törvény & \tiny C & 1 & \tiny D &\sst & \tiny \disz & & 1 & \tiny F & 1 & \tiny G &\sst & \tiny \gisz & & 1\sst & \tiny H &\sst & \tiny C \\
melodikus moll törvény & \tiny C & 1 & \tiny D &\sst & \tiny \disz & & 1 & \tiny F & 1 & \tiny G & 1 & & \tiny A & 1 & \tiny H &\sst & \tiny C \\
\end{pitemize}
\captionof{figure}{Hangnem meghatározó skálák skálatörvényei} 
\label{fig:hangnemek}

Bármely hangnem meghatározó skála jellemzői a következők
\begin{pitemize}
$-$ minden fokról indítható skála (skálafűzés), \\
$-$ minden fokra hangzati akkordok építhetőek, \\
$-$ a skála hangjai a skála bármely akkordjára játszhatóak. \\
\end{pitemize}
Látható, hogy az azonos kulcshanggal kezdődő skálák csak a harmadik és ötödik és hatodik különböznek.

\begin{practices}
\item Sorolj fel a kromatikus skála bármely hangján kezdődő dúr-, moll- és melodikus moll hangsorokat!
\item A skálákat (pl.: ,,\nameref{sec:skalamintak}'') érdemes naponta gyakorolni, mert fejlesztik az összhangot a kezek között, a pengetés pontosságát és segítenek begyakorolni az egyes hangnemekhez tartozó hangok helyét a fogólapon. Az oda-vissza játék mellett különböző minták, mint például három előre, kettő vissza is gyakorolhatók.
\end{practices}

\subsection{Alterált skálák}
Mivel az alterált skálák megértéséhez szükség van az akkordok ismeretére, a téma bővebb kifejtése az \index{alteráció},,\nameref{sec:alteracio}'' fejezetben található, az ,,\nameref{sec:akkordok}'' fejezet után.

\begin{practices}
\item Határozd meg tetszőleges dúr-, vagy moll skálák fő akkordjait!
\item Határozd meg egy hármas-, vagy négyeshangzat előfordulási helyeit a hangnem meghatározó
skálákban és az ott betöltött szerepüket (pl. G dúr a C dúr domináns akkordja).
\item Szerkessz dallamokat akkord kíséretre a funkció vertikális és horizontális szemlélete szerint!
\end{practices}

\subsection{Egyéb, nevezetes skálák}

